\documentclass{beamer}
\usepackage{etex}  %%para que funcione con Linux Ubuntu 14.04
\usepackage{beamerthemeCVC}
\usepackage{graphicx}

%% paquetes agregados por mi
%%%%%%%%%%%%%%%%%%%%%%%%%%%%%%%%%%%%%%%%%%%%%%%%%%%%%%%%%%%%%%%%%%%%%%%%%%%%%%%%
\usepackage{caption} %%paquete para hacer etiquetas caption mas pequeñas
\usepackage{subfigure} % subfiguras
%%%%%%%%%%%%%%%%%%%%%%%%%%%%%%%%%%%%%%%%%%%%%%%%%%%%%%%%%%%%%%%%%%%%%%%%%%%%%%%%%%

\usepackage[utf8]{inputenc}
\usepackage[spanish]{babel}
\usepackage{amsmath}
\usepackage{amsfonts}
\usepackage{amssymb}
\usepackage{graphicx}
\usepackage{longtable}
\usepackage{pgf-pie}
\usepackage{tikz}
\usetikzlibrary{mindmap,trees}
\usepackage{multirow}  %% multi linea de tabla


%% PRESENTATION CONFIGURATION PARAMETERS %%%%%%%%%%%%%%%%%%%%%%%%%%%%%%%%%%%%%%%
\titlebackgroundfile{templates/template_title_blue}
\framebackgroundfile{templates/template_frame_blue}
\definecolor{vermell}{HTML}{254E68}
\definecolor{gris}{HTML}{4C4C4C}
\usefonttheme{structurebold}
\setbeamercolor{author in head/foot}{fg=white}
\setbeamercolor{title in head/foot}{fg=white}
\setbeamercolor{section in head/foot}{fg=vermell}
\setbeamercolor{normal text}{fg=gris}
\setbeamercolor{frametitle}{fg=vermell}
\setbeamerfont{block title}{size={}}
\setbeamerfont{author}{size=\footnotesize}
\setbeamerfont{date}{size=\footnotesize}
\setbeamercolor{section in toc shaded}{fg=vermell} %
\setbeamercolor{section in toc}{fg=vermell} %
\setbeamercolor{caption name}{fg=gris} %??
\setbeamertemplate{itemize item}[circle]
\setbeamertemplate{itemize subitem}[circle]
\setbeamertemplate{itemize subsubitem}[circle]
\setbeamertemplate{itemize subsubsubitem}[circle]
\setbeamercolor{description item}{fg=vermell}
\setbeamercolor{description subitem}{fg=vermell}
\setbeamercolor{description subsubitem}{fg=vermell}
\setbeamercolor{description subsubsubitem}{fg=vermell}
\setbeamercolor{itemize item}{fg=vermell}
\setbeamercolor{itemize subitem}{fg=vermell}
\setbeamercolor{itemize subsubitem}{fg=vermell}
\setbeamercolor{itemize subsubsubitem}{fg=vermell}
\setbeamercolor{enumerate item}{fg=vermell}
\setbeamercolor{enumerate subitem}{fg=vermell}
\setbeamercolor{enumerate subsubitem}{fg=vermell}
\setbeamercolor{enumerate subsubsubitem}{fg=vermell}
\setbeamercolor{alerted text}{fg=vermell}
\setbeamerfont{alerted text}{series=\bfseries}
% This command makes that acrobat reader doesn't changes the colors of the slide
% when there are figures with transparencies.
\pdfpageattr {/Group << /S /Transparency /I true /CS /DeviceRGB>>}
%%%%%%%%%%%%%%%%%%%%%%%%%%%%%%%%%%%%%%%%%%%%%%%%%%%%%%%%%%%%%%%%%%%%%%%%%%%%%%%%

%      + Short title.               + Title which appears in the cover.
%      v                            v
\title[Template for slides]{Template for slides}
%       + Short author names which appear in the slides.
%       v
\author[Dennis N\'u\~nez Fern\'andez]
{   % Author names which appear in the cover page.
    Dennis N\'u\~nez  Fern\'andez%\inst{1}
}
%          + Short affiliation which appears in the slides.
%          v
\institute[UNI]
{   % Affiliation information which appears in the cover page.
    \begin{tabular}{c}
    %\inst{1}Facultad de Ingenier\'ia El\'ectrica y Electr\'onica
    Universidad Nacional de Ingenier\'ia
    \end{tabular}
}
%     + Short acronym of the conference or date of the presentation.
%     v
\date[August-2018]
{   % Conference name which appears in the cover page.
    Lima, Peru
}


\begin{document}
% Creates the cover page.
\frame{\titlepage}


\begin{frame}\frametitle{Table of Contents}
\tableofcontents
\end{frame}







\section{I. INTRODUCTION} 



\begin{frame}\frametitle{Title 1 of Introduction}

\begin{itemize}
\item Here describe the project.
\end{itemize}

\begin{figure}
\includegraphics[scale=0.4]{images/diagram.png}
\end{figure}

\end{frame}



\begin{frame}\frametitle{Title 2 of Introduction}

\begin{itemize}
\item Here describe the project.
\end{itemize}

\begin{figure}
\includegraphics[scale=0.4]{images/diagram.png}
\end{figure}

\end{frame}








\section{II. METHODOLOGY}
 

 
\begin{frame}\frametitle{Title 1 of Methodology}

\begin{itemize}
\item Here describe the project.
\end{itemize}

\begin{figure}
\includegraphics[scale=0.4]{images/diagram.png}
\end{figure}

\end{frame}



\begin{frame}\frametitle{Title 2 of Methodology}

\begin{itemize}
\item Here describe the project.
\end{itemize}

\begin{figure}
\includegraphics[scale=0.4]{images/diagram.png}
\end{figure}

\end{frame}







\section{III. SYSTEM RESULTS} 



\begin{frame}\frametitle{Title 1 of System Results}

\begin{itemize}
\item Here describe the project.
\end{itemize}

\begin{figure}
\includegraphics[scale=0.4]{images/diagram.png}
\end{figure}

\end{frame}



\begin{frame}\frametitle{Title 2}

\begin{itemize}
\item Here describe the project.
\end{itemize}

\begin{figure}
\includegraphics[scale=0.4]{images/diagram.png}
\end{figure}

\end{frame}













\section{IV. CONCLUSIONS} 



\begin{frame}\frametitle{Title 1 of Conclusions}

\begin{itemize}
\item Here describe the project.
\end{itemize}

\begin{figure}
\includegraphics[scale=0.4]{images/diagram.png}
\end{figure}

\end{frame}



\begin{frame}\frametitle{Title 2 of Conclusions}

\begin{itemize}
\item Here describe the project.
\end{itemize}

\begin{figure}
\includegraphics[scale=0.4]{images/diagram.png}
\end{figure}

\end{frame}







\end{document}

